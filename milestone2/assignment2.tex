\documentclass[12pt]{exam}
\usepackage[utf8]{inputenc}

\usepackage[margin=1in]{geometry}
\usepackage{amsmath,amssymb}
\usepackage{multicol}
\usepackage{upquote}
\usepackage{url}
\newcommand{\class}{miniHive}
\newcommand{\term}{}
\newcommand{\examnum}{Milestone 2}
\usepackage{fancyvrb}
\newcommand{\examdate}{}
\usepackage{paralist}

\pagestyle{head}
\firstpageheader{}{}{}
\runningheader{\class}{\examnum\ - Page \thepage\ of \numpages}{}
\runningheadrule



\begin{document}

\noindent
\begin{tabular*}{\textwidth}{l @{\extracolsep{\fill}} r @{\extracolsep{6pt}} l}
\textbf{\class} & \\
\textbf{\term} &&\\
\textbf{\examnum} &&\\
\textbf{\examdate} &&\\
\end{tabular*}\\
\rule[2ex]{\textwidth}{2pt}




\section*{Selection Pushing in Relational Algebra}
In relational algebra, the following equivalencies apply (among others):

\begin{align}
    \sigma_{p_1\land p_2 \land \dots\land p_n}(R) \label{split}
    &=
    \sigma_{p_1}(\sigma_{p_2}(\dots (\sigma_{p_n}(R))\dots))
    \\
        \sigma_p( \sigma_q(R)) &= \sigma_q( \sigma_p(R)) \label{push_over_sel}
    \\
    \sigma_p(R_1 \times R_2)
    &=
    \sigma_p(R_1) \times R_2 \label{push_over_cross}
    \\
    \sigma_{R_1.A_1 = R_2.A_2}(R_1 \times R_2)
    &=
    R_1 \bowtie_{R_1.A_1 = R_2.A_1} R_2 \label{create_join}
\end{align}

\noindent
Some remarks:

\begin{compactitem}
    \item 

Rule~(\ref{split}) states that a conjunction in a selection predicate may be broken into several nested selections.
At the same time, nested selections may be merged into a single selection with a conjunctive predicate.

\item
Rule~(\ref{push_over_sel}) states that nested selections may swap places.

\item
Rule~(\ref{push_over_cross}) states that a selection can be pushed down over a cross product,
if it only requires the attributes of one of the operands. In the rule as stated above,
we assume that predicate~$p$ only requires attributes from~$R_1$.
(We need to consult the data dictionary recording the name and attributes of each relation).

\item
Rule~(\ref{create_join}) describes how a selection and a cross product may be merged into a theta join,
provided that the selection predicate is a join condition. This is the case if it compares attributes of~$R_1$ and~$R_2$.
\end{compactitem}

\section{Material in Moodle}

\noindent
\begin{compactitem}
\item
The file \verb!test_raopt.py! contains unit tests. Make sure your implementation passes these tests.
\end{compactitem}

\section{What to Submit}
\begin{compactitem}

\item 
Implement rule-based selection pushing and perform it on your relational algebra queries. 
Proceed in these phases:
\begin{compactenum}
    \item Complex selection predicates are broken up, according to rule~(\ref{split}).
    \item All selections are pushed down as far as possible, according to rules~(\ref{push_over_sel}) and~(\ref{push_over_cross}).
    \item Nested selections are merged again, according to rule~(\ref{split}).
    \item Joins are introduced, where possible, according to~(\ref{create_join}).

\end{compactenum}

Note that there are more rules for the logical optimization of relational algebra, such as rules for join reordering or projection pushing. For {\em miniHive}\/, we will make do with this small set of optimization rules (for now). 

\item 
\noindent
Write a Python module \verb!raopt! that takes a relational algebra query. You may assume the query is the result of the canonical translation of SQL into relational algebra, so it uses only the operators $\sigma$, $\pi$, $\rho$ and $\times$ ). 

This is how it should work. The data dictionary~\verb!dd! contains the relational schema and can consulted during selection pushing.

\end{compactitem}

\begin{Verbatim}[frame=single,fontsize=\small]
>>> import radb.parse
>>> import raopt
>>>
>>> # The data dictionary describes the relational schema.
>>> dd = {}
>>> dd["Person"] = {"name": "string", "age": "integer", "gender": "string"}
>>> dd["Eats"] = {"name": "string", "pizza": "string"}
>>>
>>> stmt = """\project_{Person.name, Eats.pizza}
...            \select_{Person.name = Eats.name}(Person \cross Eats);"""
>>> ra = radb.parse.one_statement_from_string(stmt)
>>>
>>> ra1 = raopt.rule_break_up_selections(ra)
>>> ra2 = raopt.rule_push_down_selections(ra1, dd)
>>> ra3 = raopt.rule_merge_selections(ra2)
>>> ra4 = raopt.rule_introduce_joins(ra3)
>>>
>>> print(ra4)
\project_{Person.name, Eats.pizza} (Person \join_{Person.name = Eats.name} Eats)
\end{Verbatim}


\bigskip
\noindent
\rule{\linewidth}{0.5mm}

{\bf Remarks:}
For Milestone 2, you are not asked to find any particular optimizations to make your implementation more efficient. All you are required to provide is a correct and clean implementation.

\noindent
\rule{\linewidth}{0.5mm}


\end{document}
