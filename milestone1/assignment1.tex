\documentclass[12pt]{exam}
\usepackage[utf8]{inputenc}

\usepackage[margin=1in]{geometry}
\usepackage{amsmath,amssymb}
\usepackage{multicol}
\usepackage{upquote}
\usepackage{url}
\newcommand{\class}{miniHive}
\newcommand{\term}{}
\newcommand{\examnum}{Milestone 1}
\usepackage{fancyvrb}
\newcommand{\examdate}{}

\pagestyle{head}
\firstpageheader{}{}{}
\runningheader{\class}{\examnum\ - Page \thepage\ of \numpages}{}
\runningheadrule


\begin{document}

\noindent
\begin{tabular*}{\textwidth}{l @{\extracolsep{\fill}} r @{\extracolsep{6pt}} l}
\textbf{\class} & \\
\textbf{\term} &&\\
\textbf{\examnum} &&\\
\textbf{\examdate} &&\\
\end{tabular*}\\
\rule[2ex]{\textwidth}{2pt}

\noprintanswers

\noindent

The goal of this coding project is to build
a mini-version of Apache Hive, called {\em miniHive}\/. The first milestone is to write a query compiler that translates simple SQL queries into relational algebra. Later, we will add selection pushing to optimize the query, and then we will compile the relational algebra query into a physical query plan of MapReduce jobs.
But you are not completely on your own with this\dots

\section*{Translating SQL into Relational Algebra}

We consider simple SQL statements of the form
\begin{align*}
    &\mbox{SELECT DISTINCT } A_1, \dots, A_n\\
    &\mbox{FROM } T_1\; t_1, \dots,\; T_m t_m\\
    &\mbox{WHERE } C
\end{align*}
where
\begin{itemize}
    \item $A_1, \dots, A_n$ are attribute names,
    \item $T_1, \dots, T_m$ are  relation names, 
    \item $t_1, \dots, t_n$ are optional renamings,
    \item and $C$ is a conjunction of atomic equality conditions of the form $t_i.A = t_j.B$ or $t_i.A = c$, where $c$ is a constant,  and $A$ and $B$ are attribute names.
\end{itemize}

\noindent
The first step in {\em miniHive}\/ is to translate SQL statements into relational algebra. We make use of two existing Python modules:


\begin{itemize}
    \item 
We use \verb!sqlparse! to parse SQL statements.

More on this module at 
\url{https://github.com/andialbrecht/sqlparse}. 

\item
We use \verb!radb! to handle relational algebra statements. 

More on this module at
\url{https://github.com/junyang/radb}. 

This module requires Python~3.6!

\end{itemize}

\section{Material in Moodle}
\noindent
\begin{itemize}
\item
The file \verb!test_sql2ra.py! contains unit tests. Make sure your implementation passes these tests.
\end{itemize}

\section{What to Submit}

\noindent
Write and submit a module \verb!sql2ra! that takes a parsed SQL statement and performs the canonical translation into relational algebra, using the operators $\sigma$, $\pi$, $\times$, and $\rho$.

\noindent
This is how it should work when you spin up the interactive Python interpreter:

\begin{Verbatim}[frame=single]
>>>import sqlparse
>>>import radb
>>>import sql2ra
>>>
>>> sql = "select distinct name from person where gender='female'"
>>> stmt = sqlparse.parse(sql)[0]
>>>
>>> ra = sql2ra.translate(stmt)
>>>
>>> type(ra) # Important! Do not return a raw String, but a Project object.
<class 'radb.ast.Project'>
>>>
>>> print(ra)
\project_{name} (\select_{gender = 'female'} person)
\end{Verbatim}


\bigskip
\noindent
\rule{\linewidth}{0.5mm}


{\bf Remarks:}
For Milestone 1, you are not asked to find any particular optimizations to make your implementation more efficient. All you are required to provide is a correct and clean implementation.

\noindent
\rule{\linewidth}{0.5mm}

\end{document}
